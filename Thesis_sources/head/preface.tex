\chapter*{Preface}
\markboth{Preface}{Preface}
\addcontentsline{toc}{chapter}{Preface}

One of the aims of research in materials science is to introduce materials whose mechanical properties guarantee increasingly better performances.
A few examples can help to understand the need for high performance materials: in aerospace applications, materials of high \emph{hardness}\footnote{A glossary of the terms in italics in this paragraph referring to mechanical properties of materials is given in \autoref{app:Glossary}. A useful overview is contained in \cite{ashby2005materials}.} and low density, combined with a low melting point \cite{hamill2013hypervelocity} are needed in the making of shields able to protect spacecraft and satellites from debris traveling as fast as 15 km/s; microelectromechanical systems (or MEMS, devices whose size is in the order of $\mu$m) sometimes operate in harsh environments, for example when used as pressure sensors in turbines. In this case resistance to corrosion is important, because corrosion can lead to device failure; high \emph{strength} is also beneficial in MEMS subject to shocks, so that they don't break even when subject to intense forces \cite{schroers2007thermoplastic}; in biomedical applications, biocompatible materials with a \emph{yield strain} (so to match that of bones) and a high \emph{yield strength} are desirable for implants \cite{horton2003biomedical}. In all these cases, the ability to process the materials, so to be able to inject them into a mold and cast them into the desired shape is a big advantage, as it reduces production costs. \\

In the second half of the twentieth century, a new class of materials was introduced and developed that exhibits the properties listed above, namely \emph{metallic glasses}. The history of metallic glasses dates back to 1959, when Pol Duwez and collaborators \cite{klement1960noncrystalline} obtained flakes of an alloy of gold and silicon by means of splat quenching, a technique that allows cooling a hot melt (at above 1500 K) to room temperature at rates of $\approx 10^{7}$ K/s. Their samples, analyzed with a Debye-Scherrer camera, revealed a disordered atomic structure. This was realized via the extremely fast quenching rate, which allowed to avoid crystallization in thin samples.
Since then, other metallic alloys have been introduced which are able to avoid crystallization when quenched at relatively slow rates ($\approx 10$ K/s), retaining an amorphous microscopic structure and making it possible to obtain \emph{bulk} amorphous samples (rather than thin ones), which can be processed into the desired shape.

From a physicist's point of view it is interesting to understand how these amorphous materials form, characterize their macroscopic mechanical properties and link them to the microscopic structure of the material. 
The formation of the metallic glasses is an a example of a \emph{glass transition}, a phenomenon whose theoretical explanations haven't reached a unanimous consensus yet. As the behavior of undeformed glasses is not rationalized in a universally accepted way, one can well imagine that a thorough understanding of the behavior of metallic glasses under deformation is still out of reach.
However, several models exist nowadays which are able to reproduce some of the features shown by metallic glasses in experiments that probe their mechanical properties. They can be classified by their granularity, which is the extent to which the system of interest is subdivided into smaller parts. For instance, a macroscopic chunk of material can be thought as composed of smaller homogeneous elements (as in the finite element method \cite{vaidyanathan2001study}) or broken down into its fundamental constituents and treated as a collection of interacting atoms (as in non-equilibrium molecular dynamics simulations \cite{allen1989computer}).
Insights about the deformation behavior of metallic glasses have also come from the observation of other physical systems. An example in this sense is represented by colloidal glasses, where colloidal particles suspended in a fluid medium play the role of ``mesoscopic atoms'' whose motion (as opposed to that of real atoms) can be tracked \cite{schall2007structural} and give insights about the motion of atoms in a metallic glass. Starting from these observations one can build and improve theoretical models to help \emph{design} new amorphous materials with the desired properties.

In this thesis we model systems as composed by particles interacting via isotropic potentials and we study their behavior under a particular kind of deformation, namely cyclic shear deformation. We do so by means of computer simulation in the athermal quasi-static regime, which is supposed to describe qualitatively the phenomenology of metallic glasses well below their glass transition temperature $T_{g}$. We also compare our results with experimental and simulation data on noncolloidal suspensions, and explore similarities and differences between the two classes of systems. Our findings aim at being predictions of the outcome of oscillatory experiments on real amorphous materials, and benchmarks to test coarser-grained models.

\subsubsection*{Structure of the thesis}

In \autoref{ch:Introduction} we briefly review the remarkable properties of metallic glasses as shown by experiments. Afterwards, we outline some of the concepts that have been introduced to model their behavior. We then focus on the effects of oscillatory deformation, as studied by previous computer simulations. We finally describe the outcome of experiments and computer simulations of oscillatory deformation of dilute noncolloidal suspensions, which show how these systems can retain memory of their mechanical history. Note that \autoref{ch:Introduction} is by \emph{no means} an introduction to the vast fields of mechanical properties of materials and phenomenology and theory of glasses, but aims at providing a minimal set of conceptual tools, terminology and set the context and the motivation of the thesis. 
In \autoref{ch:ParticleModels} we describe the particle model and the algorithms used to study the deformation behavior of glassy samples of binary mixtures via computer simulation. In \autoref{ch:ParticleModelsResults} the results of our simulations are reported.
In \autoref{ch:ToyModels} we describe abstract models that retain some of key ingredients of the original particle model, while dropping significant features of it. In \autoref{ch:ToyModels} we also show that these models capture qualitatively the behavior observed in \autoref{ch:ParticleModelsResults}.\\
In \autoref{ch:Memory} the phenomenon of memory, already analyzed in the literature in the case of dilute noncolloidal suspensions, is discussed in the context of the particle models in \autoref{ch:ParticleModels} and of the models in \autoref{ch:ToyModels}. \\ 
The reader should be aware of the fact that chapters are meant to be read in a sequential way, as most of them often rely heavily on definitions and concepts illustrated in previous chapters. \\
The appendices contain the details omitted from the other chapters. \autoref{app:Gels} is a stand-alone chapter that contains the results of simulations of deformation on soft matter systems, namely \emph{particle gels} and \emph{bigels}, that use the same numerical techniques used on glasses in the core of the thesis.
	
