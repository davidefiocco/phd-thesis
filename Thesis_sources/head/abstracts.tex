%\begingroup
%\let\cleardoublepage\clearpage


% English abstract
\clearpage
\chapter*{Abstract}

In this thesis we describe the results of simulations at the atomic level of a simple model of a metallic glass under cyclic shear deformation.
We show that under oscillatory cyclic load, systems of Lennard-Jones particles exhibit a non-equilibrium transition as a function of the oscillation amplitude. At low amplitudes samples evolve at a microscopic level so to reach states which are unchanged by further oscillations, whereas above some threshold amplitude $\gamma_{c}$ they evolve indefinitely. 
Similarly to what is observed in noncolloidal suspensions, samples are able, for small oscillation amplitudes, to retain a memory of the oscillation amplitude(s). Such amplitude(s) can be subsequently read by performing additional deformation experiments.
We employ and develop simple models that are able to describe qualitatively such phenomenology, thus suggesting that a wider class of systems could be able to show the same transition and memory behavior.\\
Separately, we study by means of computer simulation the behavior under deformation of a newly found class of soft matter systems, namely bigels, and compare it with that of single-component particle gels.

\vspace{2cm}

\textbf{Keywords:} Metallic glass, mechanical properties, computer simulation, shear deformation, non-equilibrium transition, athermal quasi-static deformation, NK model, memory, aging, rejuvenation, particle gel, bigel.

% Italian abstract
\begin{otherlanguage}{italian}
\clearpage
\chapter*{Sommario}
In questa tesi sono descritti i risultati di simulazioni a livello atomico di un modello semplice di \emph{vetro metallico} sottoposto a deformazioni di taglio. Per effetto di deformazioni oscillatorie, sistemi formati da particelle interagenti tramite il potenziale di Lennard-Jones mostrano una transizione dinamica di non-equilibrio al variare dell'ampiezza della deformazione. Per piccole ampiezze, i campioni evolvono in modo tale da raggiungere stati invarianti per applicazione di deformazioni aggiuntive, mentre oltre una certa ampiezza di soglia $\gamma_{c}$ i campioni evolvono indefinitamente.
I campioni sottoposti a deformazioni di piccola ampiezza sono inoltre in grado di mantenere traccia di quest'ampiezza, analogamente a quanto osservato nel caso di sospensioni non-colloidali. Questa informazione pu\`o esser letta successivamente compiendo misure di deformazione aggiuntive.
In questa tesi sono inoltre sviluppati e utilizzati modelli teorici che catturano qualitativamente tali comportamenti. Tale accordo qualitativo suggerisce che la transizione di cui sopra e gli effetti di memoria osservati nei modelli microscopici di vetri possano estendersi ad una classe pi\`u vasta di sistemi. \\
In un'appendice al lavoro, si analizza inoltre il comportamento meccanico di una classe di sistemi di materia soffice scoperta di recente: i bigel. Le propriet\`a di deformazione di questi sistemi sono inoltre confrontate con quelle dei gel a singola componente tramite simulazione al computer.

\vspace{2cm}

\textbf{Parole chiave:} Vetro metallico, propriet\`a meccaniche, deformazioni di taglio, transizione di non-equilibrio, modello NK, memoria, aging, rejuvenation, gel, bigel. 

\end{otherlanguage}


%\endgroup			
%\vfill
